\documentclass[10pt,twocolumn]{article}

% ----- Encoding & Fonts -----
\usepackage[utf8]{inputenc}
\usepackage[T1]{fontenc}
\usepackage{lmodern}

% ----- Math & Symbols -----
\usepackage{amsmath,amssymb,amsfonts,mathtools}

% ----- Layout & Micro-typography -----
\usepackage[margin=0.8in]{geometry}
\usepackage{microtype}

% ----- Figures, Tables, Captions -----
\usepackage{graphicx}
\usepackage{booktabs}
\usepackage{siunitx}
\usepackage[font=small,labelfont=bf]{caption}

% ----- Bibliography (bioRxiv-friendly numeric) -----
\usepackage[numbers,sort&compress]{natbib}

% ----- Hyperlinks -----
\usepackage{hyperref}
\hypersetup{
  colorlinks=true,
  linkcolor=black,
  citecolor=black,
  urlcolor=black
}

% ----- Section formatting (compact) -----
\usepackage{titlesec}
\titleformat{\section}{\large\bfseries}{\thesection.}{0.5em}{}
\titleformat{\subsection}{\normalsize\bfseries}{\thesubsection}{0.5em}{}
\titleformat{\subsubsection}{\normalsize\bfseries}{\thesubsubsection}{0.5em}{}

% ----- Title & Authors -----
\title{\vspace{-0.5em}%
Casibus: Delay-Coupled Negative Feedback and Tiny Positive Coupling\\
\large Energetic and Informational Consequences for Stochastic Scheduling and Cooperation}
\author{Johannes Nagele\\Casibus Initiative (Germany)\\\texttt{jojo@casibus.de}}
\date{2025-10-27\\[0.5em]\small © 2025 Johannes Nagele — All Rights Reserved}

\begin{document}
\maketitle

\begin{abstract}
We unify two complementary results under the \emph{Casibus Principle}.
First, we show that \textbf{delay-coupled negative feedback} stabilizes stochastic flow systems without global coordination: it reduces variance and tail latencies and yields characteristic $1/f$ spectra with anti-phase cross-correlation---signatures of resonant stabilization in adaptive scheduling.
Second, we prove and validate that \textbf{tiny positive coupling} (\emph{``a small act of kindness''}) simultaneously reduces expected work $\langle W\rangle$ and increases directed information flow (transfer entropy, $\Delta \mathrm{TE}>0$) in interacting systems, with robustness across scales and estimators.
Together, these results link \emph{temporal memory as control} (via delay) with \emph{local cooperation as thermodynamic relief}, explaining how minimal, local signals can lower disorder while preserving adaptability in engineered and social systems.
\end{abstract}

\section{Introduction}
Reactive schedulers often minimize instantaneous load yet overshoot under uncertainty.
\emph{Casibus} proposes two minimal mechanisms that operate with limited information:
(i) \textbf{Delay} as an asset---a phase-shifted, negative feedback that converts random spikes into harmonic convergence; and
(ii) \textbf{Infinitesimal positive coupling} that reduces the energetic cost of stability while increasing directed predictability.

We integrate these perspectives to provide a coherent account of stability (via delay-coupled control) and adaptability/cooperation (via tiny positive coupling), relevant to GPU scheduling, multi-agent systems, and socio-technical coordination.

\section{Unified Framework: Casibus}
\subsection{Negative (Stabilizing) Arm: Delay-Coupled Control}
For $N$ units with queue $q_i(t)$, define the score
\begin{equation}
s_i(t) = a\,q_i(t) + b\,\mathrm{Var}(q_i) + c\,\max\!\bigl(0,\,\dot q_i(t-\tau)\bigr),
\end{equation}
and assign $\mathrm{unit}^*(t) = \arg\min_i s_i(t)$.
A nonzero delay $\tau$ induces anti-phase coupling that damps oscillations and yields $1/f^\alpha$ spectra ($\alpha\!\approx\!1$).

\subsection{Positive (Cooperative) Arm: Tiny Acts of Kindness}
Consider stable linear interactions $x_{t+1}=Ax_t+\xi_t$ with $A=aI+K$, $\|\lambda_i(A)\|<1$, $\xi_t\sim\mathcal N(0,\Sigma_\xi)$.
A small positive increment $\delta k>0$ on an edge reduces long-run expected work $\langle W\rangle$ and increases transfer entropy along that edge.
Agent-based public-goods dynamics show that transient cooperative impulses diffuse and create persistent cooperation shifts.

\section{Models and Methods}
\subsection{Scheduling Simulator (Discrete Time)}
We compare \emph{Random}, \emph{Round Robin}, and \emph{Casibus} policies under Poisson arrivals and unit service.
Metrics: mean and 95th percentile queue length, utilization variance $\mathrm{Var}(U)$, power spectral density (PSD), and cross-correlation.

\subsection{Stochastic Linear Networks (AR)}
Stationary covariance $\Sigma$ solves $\Sigma=A\Sigma A^\top+\Sigma_\xi$.
We measure instantaneous work $W_t=\|x_{t+1}-x_t\|^2$ and its average $\langle W\rangle$.

\subsection{Information-Theoretic Metrics}
Transfer entropy (TE)
\[
\mathrm{TE}_{X\to Y}=\mathbb E\!\left[\log\frac{p(y_t|y_{t-1},x_{t-1})}{p(y_t|y_{t-1})}\right]
\]
estimated via (i) discretized plug-in with bias correction and (ii) $k$-NN (Kraskov-type).
Differential entropy via $k$-NN; CIs via block bootstrap.

\subsection{Game-Theoretic Simulation}
Ring-structured public-goods game with softmax decisions and diffusive adoption of local utility multipliers.
Kindness pulses provide temporary local gain that can spread via diffusion.

\subsection{Statistical Testing}
Primary contrasts over $\delta k\in\{-0.1,0,+0.1\}$:
$\Delta \mathrm{TE}$, $\Delta \langle W\rangle$, $\Delta H$.
We report $p$-values (permutation/t-tests), Hedges' $g$, and 95\% CIs.
Trajectory-level comparison in the game setting via AUC permutation.

\section{Results}
\subsection{Delay-Coupled Control in Scheduling}
\begin{figure}[t]
\centering
\includegraphics[width=\linewidth]{figures/fig1_psd_comparison.pdf}
\caption{PSD (log--log): Casibus exhibits $\sim 1/f^\alpha$ with $\alpha\approx 1$.}
\end{figure}

\begin{figure}[t]
\centering
\includegraphics[width=\linewidth]{figures/fig2_crosscorr_heatmap.pdf}
\caption{Cross-correlation $R_{su}(\ell)$ across delays $\tau$: negative valleys at positive lags confirm anti-phase resonance.}
\end{figure}

\begin{figure}[t]
\centering
\includegraphics[width=\linewidth]{figures/fig3_phase_plane.pdf}
\caption{Phase plane $q$ vs $\dot q$: stable spiral attractor under delay-coupled feedback.}
\end{figure}

\begin{figure}[t]
\centering
\includegraphics[width=\linewidth]{figures/fig4_tau_variance.pdf}
\caption{Variance vs delay $\tau$: stability optimum at moderate delays.}
\end{figure}

\begin{table}[t]
\centering
\begin{tabular}{lcccc}
\toprule
Policy & Mean & p95 & Var(U) & Notes\\
\midrule
Random & 1.00 & 1.00 & 1.00 & baseline\\
Round Robin & 0.88 & 0.92 & 0.93 & periodic\\
Casibus & \textbf{0.62} & \textbf{0.64} & \textbf{0.72} & resonant\\
\bottomrule
\end{tabular}
\caption{Normalized metrics (synthetic): Casibus lowers mean/tail latency and utilization variance.}
\end{table}

\subsection{Energetics \& Information Flow under Tiny Positive Coupling}
\begin{figure}[t]
\centering
\includegraphics[width=.98\linewidth]{fig_r1_dW.png}
\caption{$\Delta \langle W\rangle$ across $\delta k\in\{-0.1,0,+0.1\}$. Error bars: 95\% CI.}
\end{figure}

\begin{figure}[t]
\centering
\includegraphics[width=.98\linewidth]{fig_r1_dTE.png}
\caption{$\Delta \mathrm{TE}$ on the manipulated edge. Error bars: 95\% CI.}
\end{figure}

\begin{figure}[t]
\centering
\includegraphics[width=.98\linewidth]{fig_r2_heat_te.png}
\caption{Proportion of runs with $\Delta \mathrm{TE}>0$ across system size $n$ and density.}
\end{figure}

\begin{figure}[t]
\centering
\includegraphics[width=.98\linewidth]{fig_r2_heat_work.png}
\caption{Proportion of runs with $\Delta \langle W\rangle<0$ across $n$ and density.}
\end{figure}

\begin{figure}[t]
\centering
\includegraphics[width=.98\linewidth]{fig_r3_knn_vs_binning.png}
\caption{Estimator robustness: discretized vs $k$-NN TE.}
\end{figure}

\begin{figure}[t]
\centering
\includegraphics[width=.98\linewidth]{fig_r4_te_vs_h.png}
\caption{Correlation between $\Delta \mathrm{TE}$ and $\Delta H$ on target nodes.}
\end{figure}

\begin{figure}[t]
\centering
\includegraphics[width=.98\linewidth]{fig_r5_auc_perm.png}
\caption{Trajectory-level AUC permutation test in the game setting.}
\end{figure}

\section{Analytical Approximation \& Unified Theory}
\subsection{Frequency Response of Delay-Coupled Control}
Linearizing small fluctuations yields an effective response $H(\omega)$ with magnitude $\propto 1/\omega^\alpha$ when the delay term balances local trend and variance, explaining $\alpha\approx 1$ and anti-phase cross-correlation.

\subsection{Energetic--Informational Link for Positive Coupling}
Transfer entropy is a functional over conditional kernels. The Gâteaux derivative and second variation relate to Fisher information, clarifying why histogram binning lacks a classical $dx$ limit and instead lives on the manifold of probability measures.
Under a small $\delta k>0$, directed predictability increases while required work decreases.

\section{Discussion}
Delay is not a nuisance but a \emph{control primitive}; tiny cooperative nudges are \emph{thermodynamic reliefs}.
Casibus combines both: temporal memory stabilizes flows; minimal positive coupling lowers local disorder and increases directed information flow.
Applications span GPU job routing, multi-agent coordination, customer operations, and social cooperation dynamics.

\subsection{Limitations}
Estimator bias at small samples; model idealizations in AR networks and game rules; synthetic normalization in scheduling metrics. Future work: closed-form stability bounds, hardware validation on GPU clusters, causal TE estimation under finite data.

\section{Conclusion}
Minimal information, minimal cooperation, and minimal delay suffice to stabilize queues, reduce energetic cost, and increase directed predictability.
Casibus provides a compact recipe for robust, adaptive coordination across engineered and social domains.

\section*{Appendix C: Differential vs Information Geometry (Sketch)}
Transfer entropy functional derivatives and their Fisher-information links provide geometric intuition for why coarse-graining lacks a classical differential limit, reinforcing the information-geometric stance for inference under finite samples.

% ----- Bibliography -----
\bibliographystyle{unsrtnat} % numeric, unsorted by appearance acceptable at bioRxiv/Zenodo
\bibliography{casibus_core,atk_refs}

\end{document}
